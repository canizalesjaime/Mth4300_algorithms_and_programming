\documentclass[a4paper]{article}

\usepackage{fullpage} % Package to use full page
\usepackage{parskip} % Package to tweak paragraph skipping
\usepackage{tikz} % Package for drawing
\usepackage{amsmath}
\usepackage{hyperref}

\title{MTH 4300: Algorithms, Computers, and Programming II}
\author{Fall 2024}
\date{Midterm 2 Review}

\begin{document}
\maketitle


\section{TRUE OR FALSE}
\begin{enumerate}
    \item  Returning by reference allows you to avoid copying an object, 
           improving performance, but returning by value ensures the caller
           receives a new copy of the object.
    \item The this pointer in C++ is a constant pointer that holds the address
          of the current object.
    \item The size of an std::vector is fixed once it is created and cannot change.
    \item std::list in C++ allows random access to its elements, just like std::vector.
    \item You can overload the arrow (-$>$) operator in a class, and it is commonly
          used when a class contains or behaves like a pointer.
    \item In a singly linked list, each node has a pointer to the previous node as well as the next node.
    \item The time complexity of the selection sort algorithm is O(n log n) in the worst case.
    \item The fstream class in C++ allows for both input and output operations on files.
    \item Overloaded operators must have at least one operand that is of user-defined type(a class).
    \item (*ptr).method() is the same as ptr-$>$method()  
\end{enumerate}
\newpage


\section{SHORT ANSWER}
\begin{enumerate}
    \item How do you declare a member function that guarantees it will not modify the object it belongs to?
    \item What happens to the elements of a vector when it resizes after exceeding its current capacity?
    \item What is an advantage of using std::list over std::vector?
    \item In stl the list stl data structure has a method named push\_front() 
          that adds an element to the front of the list. What is the time complexity of this method?
    \item What operator do you have to overload as friend function(typically)?
    \item What class in fstream is used to only open files?
    \item When is a destructor called?
    \item Given the files \texttt{main.cpp myclass1.cpp myclass1.h} how would you compile these in a terminal
    \item Write any function signature that uses default parameters(arguments).
    \item How does the selection sort algorithm determine which element to swap at each step?  
\end{enumerate}
\newpage 


\section{CODING}
\begin{enumerate}
    \item Write the code for the method(adds a node to the end of a linked list. Return true if successful otherwise false. Do not use stl): 
    \begin{verbatim}
        bool LinkedList::push_back(int val)
        {
            ...
        }
    \end{verbatim}\newpage
    \item The code below has more than 5 errors. Find at least 5 for full credit!\\
    \textbf{rectangle.h:}
    \begin{verbatim}
#ifndef RECTANGLE_H
#define RECTANGLE_H

class Rectangle {
public:
    Rectangle(double width, double height);  // Constructor                 
    double getPerimeter() const;             // Member function to get perimeter
    void setHeight(double height) const;     // Sets height
    void getHeight();                        // Gets height


private:
    double width;                            // Member variables
    double height;
};

#endif // RECTANGLE_H
    \end{verbatim}
    \textbf{rectangle.cpp:}
    \begin{verbatim}
#include<iostream>
#include "rectangle.h"

// Constructor definition
Rectangle::Rectangle(double width, double height) : width(width), height(height) {}

// Function to calculate the area of the rectangle
double Rectangle::getArea() const 
{
    return width * height;
}

// Sets height
double Rectangle::getPerimeter() const
{
    return 2 * (width + height);
}

// Function to calculate the perimeter of the rectangle
double Rectangle::setHeight(double h) const
{
    height=h;
}

// Gets height
void getHeight()
{
    return height;
}                      
\end{verbatim} 
\newpage
\textbf{main.cpp:}
\begin{verbatim}
#include <iostream>

int main() 
{
    Rectangle rect(10.0, 5.0);  // Create a Rectangle object with width 5.0 and height 3.0
    std::cout << "Area: " << rect.getArea() << std::endl;
    std::cout << "Perimeter: " << rect.getPerimeter() << std::endl;
    std::cout <<rect<< std::endl;
    return 0;
}
\end{verbatim}
\end{enumerate}
\newpage


\section{SOLUTIONS}
WILL BE RELEASED ON THURSDAY AFTER WE COVER THE QUESTIONS IN CLASS
\end{document}