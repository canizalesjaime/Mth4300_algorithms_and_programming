\documentclass[a4paper]{article}

\usepackage{fullpage} % Package to use full page
\usepackage{parskip} % Package to tweak paragraph skipping
\usepackage{tikz} % Package for drawing
\usepackage{amsmath}
\usepackage{hyperref}

\title{MTH 4300: Algorithms, Computers, and Programming II}
\author{HW \#3}
\date{Due Date: October 31st, 2024}

\begin{document}
\maketitle


\section*{Problem 1}
You need to create a class called \texttt{Book} that represents a book in a library. The class should include the following requirements and make use of the C++ concepts listed below:

\begin{itemize}
    \item \textbf{Data Members}:
    \begin{itemize}
        \item Private data members:
        \begin{itemize}
            \item \texttt{title} of type \texttt{std::string}
            \item \texttt{author} of type \texttt{std::string}
            \item \texttt{yearPublished} of type \texttt{int}
            \item \texttt{price} of type \texttt{double}
        \end{itemize}
    \end{itemize}

    \item \textbf{Constructor}:
    \begin{itemize}
        \item The class should have a constructor that takes the following parameters:
        \begin{itemize}
            \item \texttt{bookTitle} (a \texttt{std::string} passed by reference) for the title of the book
            \item \texttt{bookAuthor} (a \texttt{std::string} passed by reference) for the author of the book
            \item \texttt{publishedYear} (an integer with a default value of \texttt{1900}) for the year the book was published
            \item \texttt{bookPrice} (a double with a default value of \texttt{0.0}) for the price of the book
        \end{itemize}
        \item Use an \textbf{initialization list} to initialize all the data members.
    \end{itemize}

    \item \textbf{Methods}:
    \begin{itemize}
        \item Implement a method called \texttt{applyDiscount()} that takes a \texttt{double} discount percentage by \textbf{reference} and applies it to the price of the book.
        \item Implement a method called \texttt{getBookInfo()} that returns the book's details (title, author, year published, and price) as a formatted string. This method should be marked as \textbf{const} since it does not modify the object's state.
    \end{itemize}
\end{itemize}

\subsection*{Example Usage}

\begin{verbatim}
    string bookName="The Great Gatsby";
    string author="F. Scott Fitzgerald";
    Book myBook(bookName, author, 1925, 15.99);
    double discount = 10.0; // 10% discount
    myBook.applyDiscount(discount);
    myBook.getBookInfo();
\end{verbatim}

\subsection*{Implementation Steps}

\begin{itemize}
    \item Define the \texttt{Book} class with the required private data members.
    \item Implement the constructor using an \textbf{initialization list} with default arguments.
    \item Implement the \texttt{applyDiscount()} method using \textbf{pass-by-reference} for the discount parameter.
    \item Implement the \texttt{getBookInfo()} method, ensuring it is marked as a \textbf{const} member function.
\end{itemize}

\subsection*{Your Task}
Write the full implementation of the \texttt{Book} class according to the above specifications.
\newpage

\section*{Problem 2}
Create a class for a 3 by 3 matrix(using arrays and not vectors) named \texttt{Matrix33}:
\begin{itemize}
    \item Make sure the private attribute is a 2d array \texttt{double matrix[3][3];}
    \item A constructor that accepts a 2d array as an input parameter
    \item Add a default constructor that takes no arguments and does nothing in the body: \begin{verbatim}
        matrix33(){}
    \end{verbatim} 
    \item Overload * operator for matrix multiplication
    \item Overload * operator for scalar multiplication
    \item Overload + operator for matrix addition
    \item Overload $<<$ operator to print matrix
    \item Overload $>>$ operator, and prompt user to enter 9 consecutive values
    \item Write a function to compute the determinant of the matrix 
\end{itemize}
\newpage

\section*{Problem 3}
\begin{itemize}
    \item Separate the interface and implementation for problem 2.
    \item Modify the 3d\_point.cpp file we went over in class
    (you can find it in week9$\backslash$more\_object\_oriented\_stuff-10\_22 repo), 
    by separating the interface and implemetation, then rename the class \texttt{Vector3}.
    \item Create a separate main.cpp file where you include the headers for \texttt{Matrix33} and \texttt{Vector3}
    \item Overload the operator (), for accessing the private attributes of the \texttt{Vector3} and \texttt{Matrix33} classes.
    \begin{verbatim}
        double operator()(int row, int col)
        {
            return matrix[row][col];
        }
    \end{verbatim} 
    \item Write a function in main.cpp that takes a  \texttt{Matrix33}=A and \texttt{Vector3}=x as input parameters
          and computes Ax=b, and returns a type of \texttt{Vector3}(b). 
    \item prompt the user to enter a matrix(3 by 3) and vector(3), then call your function to compute the product, then print the result.
\end{itemize}
\newpage
 

\section*{Problem 4}
Do problem 2, but for an n by m matrix using the vector template class. 
In the constructor add the parameters for the number of rows (n) and number of columns(m).


\end{document}