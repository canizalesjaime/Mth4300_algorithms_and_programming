\documentclass[a4paper]{article}

\usepackage{fullpage} % Package to use full page
\usepackage{parskip} % Package to tweak paragraph skipping
\usepackage{tikz} % Package for drawing
\usepackage{amsmath}
\usepackage{hyperref}

\title{MTH 4300: Algorithms, Computers, and Programming II}
\author{Fall 2024}
\date{Course Number: 56474; Section: KTRA}

\begin{document}
\maketitle

\textbf{Instructor:} Jaime Canizales\\\\
\textbf{Email:} jaime.canizales@hunter.cuny.edu (will update to baruch email soon)\\\\
\textbf{Office Hours:} By appointment. \\\\
\textbf{Recommended Texts:} S. Lippman, J. Lajoie, B. Moo, C++ Primer, 5th Edition, Addison-Wesley
Professional, 2012 and B. Stroustrup, The C++ Programming Language, 4th Edition, Addison-Wesley
Professional, 2013.\\\\
\textbf{Meeting Time \& Location:} Tuesdays and Thursdays 2:55PM - 4:35PM, VC 6-125.\\\\
\textbf{Prerequisite:} MTH 3300 or CIS 2300, as well as at least one class in Calculus (MTH 2205, 2206, 2207,
2610, 2630, 3010). Note: this class is not open to those who have credit for CIS 3100 or
CIS 4100. If you have credit for CIS 3100 and have not already been in contact with me
about this, please see me immediately.\\\\
\textbf{Software/Technology:} A C++ compiler. \\\\
\textbf{Learning Goals:} Upon completion of this course students will be able to:
\begin{itemize}
    \item make use of pointers;
    \item create classes (abstract data types);
    \item create constructors and destructors;
    \item write class methods;
    \item overload functions and operations (polymorphism);
    \item understand the notion and the implementation of inheritance;
    \item properly implement portions of the C++ Standard Template Library;
    \item and solve problems efficiently by constructing and implementing appropriate algorithms and
    data structures such as lists, stacks, and binary search trees.
\end{itemize}
This syllabus is likely to evolve as the term progresses.
\begin{center}
    \begin{tabular}{||c c c ||} 
     \hline
     Class Number & Date & Topics  \\ [0.5ex] 
     \hline\hline
     1 & 8/29 & Intro  \\ 
     \hline
     2 & 9/3 & Basics of C++  \\
     \hline
     3 & 9/5 & Control Flow in C++  \\
     \hline
     4 & 9/10 & Functions  \\
     \hline
     5 & 9/12 & Recursion  \\ [1ex] 
     \hline
    \end{tabular}
    \end{center}

\newpage
\textbf{Course Policy:}
\begin{enumerate}
    \item Attendance: You are expected to attend every class. Please arrive promptly at the beginning of
    class.
    \item Classroom Demeanor: During class meetings, it is expected that you respect the class and your
    classmates. In particular, you should refrain from making insensitive remarks, talking at a
    disruptive volume, or spending time using phones or computers on non-class related activities.
    \item Participation: The best way to learn programming is by doing, and you will spend a significant
    amount of class time each period reading or writing programs on your own. I will frequently
    pause class to ask you to try a problem or consider a question. When I do so, I expect you to try
    to discuss approaches, ask questions, or explain things to one another. I expect you to take these
    moments seriously. Thank you in advance for your cooperation
    \item Assignments: Homework assignments will be assigned on Brightspace frequently. In addition,
    there will be some suggested exercises, not to be turned in.
    \item Exams: There will be 2 midterms and a final (consult the class schedule for specific dates).
    According to department policy, any student who scores less than 50\% on the final exam will
    not receive a passing grade for the course. The midterms and final exam will be given
    during in-person meetings.
    \item Grading:
    Assignments = 15\% \\
    Midterms(2) = 52\% \\ 
    Final Exam = 33\%
    \item Grading Scale: A 93.00, A- 90.00, B+ 87.00, B 83.00, B- 80.00, C+ 77.00, C 73.00, C- 70.00, D+
    67.00, D 60.00, F < 60. I don’t intend to alter this scale, but I reserve the right to do so; if I do,
    it will almost certainly be in your favor.
    \item Late Assignment Policy: For full credit, all assignments should be submitted by the due date, by
    11:59 PM. I will take off 10 points from the assignment for each day late, but I will give you a
    total of three penalty-free days late throughout the term (for example, you could submit one
    assignment two days late, and another assignment one day late, without penalty; further late
    submissions would then be penalized at 10 points per day).
    \item Accommodations: Baruch has a continuing commitment to providing reasonable
    accommodations for students with disabilities. Like so many things this term, the need for
    accommodations and the process for arranging them have been altered by COVID-19 and the
    safety protocols currently in place. Students with disabilities who may need some
    accommodation in order to fully participate in this class should contact Student Disability
    Services as soon as possible at disability.services@baruch.cuny.edu.
    \item On Getting Help, and Artificial Intelligence: Feel free to come to office hours or email me at any
    time if you need assistance!
    For programming assignments (and quizzes and exams), copying of any code produced by
    ChatGPT, similar artificial intelligence, or other human or internet sources is prohibited.
    “Paraphrasing” of code, whereby code produced by artificial intelligence or another source is
    lightly changed, and then submitted as original, is also prohibited. Submitting such code
    without acknowledgment is an academic integrity violation.
    2
    Generally, I caution students to be very careful when getting help – certainly from Generative
    AI, but also from videos, from online sources, and even from friends and tutors. This is not a
    discouragement from using any of these resources, but it is crucial that you actively engage with
    them, rather than simply listen or read: take notes, ask questions, pause and try variations on
    your own, etc. As an experienced instructor, I have seen student after student fall into the
    following trap:
    - student gets help from a person or source that presents a solution to a problem, without much
    input from the student themself;
    - then, due to the clarity of the presentation, student feels like they understand the solution;
    - finally, student finds that they are unable to make any headway with closely-related problems,
    or even with reproducing the solution to the original problem.
    The middle step, the sensation of comprehension, can be very misleading. (And the last step
    typically happens during exams.)
    Generative AI is a neat tool for exploration, but it makes the trap exceptionally easy to fall into.
    While limited use of AI is not prohibited for problem sets or studying, you are discouraged from
    using it - especially since the accuracy of its responses, while impressive, is far from perfect.
    There are ways in which AI can be used constructively for education, but if your use does not
    involve substantial quantities of code that you have written on your own, then you are at risk of
    getting zero or negative return from your “studying.”
    \item Collaboration: You are encouraged to think about how to proceed on programming assignments
    and projects with your colleagues. However, when you sit down to code them, it is expected that
    what you write will be entirely your own work. You are not to copy (or “closely paraphrase”)
    your classmates work.
    Additionally, the following rule absolutely must be followed for coding assignments:
    You are to report sources and people that you consulted in writing your code (again,
    copying is NOT acceptable - this is meant for people and sources/websites which gave
    you hints or inspiration).
    Violation of this rule will result in a report of academic dishonesty being sent to the Office of the
    Dean of Students.
    \item Academic Honesty: The Department of Mathematics fully supports Baruch College's policy on
    Academic Honesty which states, in part: “Academic dishonesty is unacceptable and will not be
    tolerated. Cheating, forgery, plagiarism and collusion in dishonest acts undermine the college's
    educational mission and the students personal and intellectual growth. Baruch students are
    expected to bear individual responsibility for their work, to learn the rules and definitions that
    underlie the practice of academic integrity, and to uphold its ideals. Ignorance of the rules is not
    an acceptable excuse for disobeying them. Any student who attempts to compromise or devalue
    the academic process will be sanctioned.”
    Academic sanctions in this class will range from an F on the assignment to an F in this course.
    A report of suspected academic dishonesty will be sent to the Office of the Dean of Students.
    Additional information and definitions can be found at
    http://www.baruch.cuny.edu/academic/academic\_honesty.html.
\end{enumerate}
\end{document}